\documentclass{acmart}
%Title
\title{Bag of words with Bayesian Network to detect languages}
%Authors
\author{Santiago E. Bocel}
\email{santiagobocel10@gmail.com}
\author{Brenner Hernandez}
\email{velasquezbrenner@gmail.com}
\author{Pablo Muralles}
\email{pablomuralles28@gmail.com}
\author{Roberto Solares}
\email{betosolaresgar@gmail.com}
\affiliation{%
    \institution{Universidad Rafael Landívar}
    \postcode{01016}
    \city{Guatemala City}
    \country{Guatemala}}

% keywords
\keywords{bag of words, bayesian network}

% concepts
\begin{CCSXML}
<ccs2012>
   <concept>
       <concept_id>10010147.10010257.10010293.10010300.10010306</concept_id>
       <concept_desc>Computing methodologies~Bayesian network models</concept_desc>
       <concept_significance>500</concept_significance>
       </concept>
 </ccs2012>
\end{CCSXML}
\ccsdesc[500]{Computing methodologies~Bayesian network models}

\begin{document}

    % ABSTRACT
    \begin{abstract}
    UPDATED---\today. ABSTRACT...
    \end{abstract}

    \maketitle

    \section{Introduction}
    Text classification is one of the applications of Machine Learning and consists of cataloging the texts based on their content, that is, performing an analysis of the words to decide what type of text is being identified.
    \newline
    \newline
    This work is ideal for a machine as they are ideal for processing large amounts of information. However, since the machine does not initially know how to catalog a text based on any criteria, it requires a learning process in advance.
    \newline
    
    
    

    \section{Section 1}
    información...

    \section{Section 2}
    más información...

    % REFERENCES
    \begin{thebibliography}{9}

    \bibitem{latexcompanion} 
    Michel Goossens, Frank Mittelbach, and Alexander Samarin. 
    \textit{The \LaTeX\ Companion}. 
    Addison-Wesley, Reading, Massachusetts, 1993.

    \bibitem{einstein} 
    Albert Einstein. 
    \textit{Zur Elektrodynamik bewegter K{\"o}rper}. (German) 
    [\textit{On the electrodynamics of moving bodies}]. 
    Annalen der Physik, 322(10):891–921, 1905.

    \bibitem{knuthwebsite} 
    Knuth: Computers and Typesetting,
    \texttt{http://www-cs-faculty.stanford.edu/\~{}uno/abcde.html}
    \end{thebibliography}
\end{document}

