\documentclass[sigconf,12pt,review=false,natbib=false]{acmart}

\usepackage[style=ACM-Reference-Format,backend=bibtex,sorting=nty]{biblatex}
\addbibresource{research.bib}

\begin{document}

% ACM Format
\settopmatter{printacmref=false}
\setcopyright{none}
\renewcommand\footnotetextcopyrightpermission[1]{}
\pagestyle{plain}

%Title
\title{Bag of words with Bayesian Network to detect languages}

%Authors
\settopmatter{authorsperrow=2}

\author{Santiago E. Bocel}
\affiliation{%
    \institution{Universidad Rafael Landívar}
    \postcode{01016}
    \city{Guatemala City}
    \country{Guatemala}}
\email{santiagobocel10@gmail.com}

\author{Roberto Solares}
\affiliation{%
    \institution{Universidad Rafael Landívar}
    \postcode{01016}
    \city{Guatemala City}
    \country{Guatemala}}
\email{betosolaresgar@gmail.com}

\author{Brenner Hernandez}
\affiliation{%
    \institution{Universidad Rafael Landívar}
    \postcode{01016}
    \city{Guatemala City}
    \country{Guatemala}}
\email{velasquezbrenner@gmail.com}

\author{Pablo Muralles}
\affiliation{%
    \institution{Universidad Rafael Landívar}
    \postcode{01016}
    \city{Guatemala City}
    \country{Guatemala}}
\email{pablomuralles28@gmail.com}

% abstract
\begin{abstract}

Text classification also known as text tagging or text categorization is the process of categorizing
text into organized groups. By using Natural Language Processing (NLP), text classifiers can
automatically analyze text and then assign a set of pre-defined tags or categories based on its
content. \\

There are different types of text classifiers, for example, language detection, process automation,
virtual legislation, sentiment detection, etc. That is why the classification of texts is becoming
an increasingly important tool, since it allows us to obtain information from data and make use of
them quickly, something that is very important in the information age. \\

On the other hand, machine learning in its most basic form is the practice of using algorithms to
analyze data, learn from it, and then make a determination or prediction about something in the
world. Where there are endless techniques for learning, representation and optimization. \\

It is for these reasons that the combination of machine learning with text classification is a very
powerful but at the same time very complex tool and a field in which there is still much to
explore. \\

\end{abstract}

\maketitle

\section{Introduction}

Text classification is one of the applications of Machine Learning and consists of cataloging the
texts based on their content, that is, performing an analysis of the words to decide what type of
text is being identified. \\

This work is ideal for a machine as they are ideal for processing large amounts of information.
However, since the machine does not initially know how to catalog a text based on any criteria, it
requires a learning process in advance. \\

In this research, the Bag of Words (BOW) model will be used, which is a method used in language
processing to classify words according to the tag. \\

Many language processing tasks involve a classification, in which different machine learning methods
can be used such as Maximum Entropy (ME), Support Vector Machines (SVE), Naive Bayes (NB) and many
more, that is why that in this work the Naive Bayes algorithm was used, more specifically the
Gaussian Naive Bayes algorithm. The Gaussian Naive Bayes algorithm is of great help as it proposes a
solution to the categorization of text by assigning a label or a category to an entire text or
document. \\

All these methods and tools can be applied in different types of classifiers such as the
classification of feelings, which is the process of automating or identifying opinions in the text
and labeling them as positive, negative or neutral, based on the emotions or labels that it possesses.
each one, the spam classification of some text, the automatic generation of subtitles, among others.
In the case of this research, we focus on knowing the language in which a text is written, language
recognition. \\

This paper is structured as follows. The Fundamentals section describes the prior knowledge that the
reader is recommended to possess in order to have a better understanding of the research. In the
section The problem, the problem to be solved with this investigation is detailed as well as its
requirements. The Similar Implementations section describes which experiments and projects have
solved the same problem and how they have done it. In the section Our Solution, it is explained in
detail how the solution was carried out, as well as why certain decisions were made and what problems
were encountered when carrying out the same. The Results section describes what information could be
obtained both in the training phase as well as in the experimentation and testing phase. Finally, the
Conclusions section expresses the thoughts and observations of the authors after conducting the
research. \\

\section{Fundamentals}

\section{Problem}

\section{Similar Implementations}

\section{Our Solution}

\section{Results}

\section{Conclusions}

% References
\nocite{*}
\printbibliography

\end{document}
